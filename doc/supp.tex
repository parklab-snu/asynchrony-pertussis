\documentclass[12pt]{article}
% \usepackage[top=1in,left=1in, right = 1in, footskip=1in]{geometry}
\usepackage[top=1in,footskip=1in]{geometry}

\usepackage{graphicx}
\usepackage{xspace}
%\usepackage{adjustbox}

\usepackage{multirow}
\usepackage{booktabs}

\usepackage{pdflscape}

\usepackage{grffile}

\newcommand{\comment}{\showcomment}
%% \newcommand{\comment}{\nocomment}

\newcommand{\showcomment}[3]{\textcolor{#1}{\textbf{[#2: }\textsl{#3}\textbf{]}}}
\newcommand{\nocomment}[3]{}

\newcommand{\jd}[1]{\comment{cyan}{JD}{#1}}
\newcommand{\swp}[1]{\comment{magenta}{SWP}{#1}}
\newcommand{\bmb}[1]{\comment{blue}{BMB}{#1}}
\newcommand{\djde}[1]{\comment{red}{DJDE}{#1}}

\newcommand{\eref}[1]{Eq.~(\ref{eq:#1})}
\newcommand{\fref}[1]{Fig.~\ref{fig:#1}}
\newcommand{\Fref}[1]{Fig.~\ref{fig:#1}}
\newcommand{\sref}[1]{Sec.~\ref{#1}}
\newcommand{\frange}[2]{Fig.~\ref{fig:#1}--\ref{fig:#2}}
\newcommand{\tref}[1]{Table~\ref{tab:#1}}
\newcommand{\tlab}[1]{\label{tab:#1}}
\newcommand{\seminar}{SE\mbox{$^m$}I\mbox{$^n$}R}

\usepackage{amsthm}
\usepackage{amsmath}
\usepackage{amssymb}
\usepackage{amsfonts}
\usepackage[utf8]{inputenc} % make sure fancy dashes etc. don't get dropped

\usepackage{lineno}
\linenumbers

\usepackage[pdfencoding=auto, psdextra]{hyperref}

\usepackage{natbib}
\bibliographystyle{unsrt}
\date{\today}

\usepackage{xspace}
\newcommand*{\ie}{i.e.\@\xspace}

\usepackage{color}

\newcommand{\Rx}[1]{\ensuremath{{\mathcal R}_{#1}}\xspace} 
\newcommand{\RR}{\ensuremath{{\mathcal R}}\xspace}
\newcommand{\Rres}{\Rx{\mathrm{res}}}
\newcommand{\Rinv}{\Rx{\mathrm{inv}}}
\newcommand{\Rhat}{\ensuremath{{\hat\RR}}}
\newcommand{\Rt}{\ensuremath{{\mathcal R}(t)}\xspace}
\newcommand{\tsub}[2]{#1_{{\textrm{\tiny #2}}}}
\newcommand{\dd}[1]{\ensuremath{\, \mathrm{d}#1}}
\newcommand{\dtau}{\dd{\tau}}
\newcommand{\dx}{\dd{x}}
\newcommand{\dsigma}{\dd{\sigma}}

\newcommand{\rx}[1]{\ensuremath{{r}_{#1}}\xspace} 
\newcommand{\rres}{\rx{\mathrm{res}}}
\newcommand{\rinv}{\rx{\mathrm{inv}}}

\newcommand{\psymp}{\ensuremath{p}} %% primary symptom time
\newcommand{\ssymp}{\ensuremath{s}} %% secondary symptom time
\newcommand{\pinf}{\ensuremath{\alpha_1}} %% primary infection time
\newcommand{\sinf}{\ensuremath{\alpha_2}} %% secondary infection time

\newcommand{\psize}{{\mathcal P}} %% primary cohort size
\newcommand{\ssize}{{\mathcal S}} %% secondary cohort size

\newcommand{\gtime}{\tau_{\rm g}} %% generation interval
\newcommand{\gdist}{g} %% generation-interval distribution
\newcommand{\idist}{\ell} %% incubation-period distribution

\newcommand{\total}{{\mathcal T}} %% total number of serial intervals

\usepackage{lettrine}

\newcommand{\dropcapfont}{\fontfamily{lmss}\bfseries\fontsize{26pt}{28pt}\selectfont}
\newcommand{\dropcap}[1]{\lettrine[lines=2,lraise=0.05,findent=0.1em, nindent=0em]{{\dropcapfont{#1}}}{}}

\begin{document}

\begin{flushleft}{
	\Large
	\textbf\newline{
    Supplementary Materials for: Delayed introduction and susceptible variation drive spatial asynchrony in pertussis epidemics
	}
}
\newline
\\
Sang Woo Park\textsuperscript{1,*}
\\
\bigskip
\textbf{1} School of Biological Sciences, Seoul National University, Seoul, Korea
\bigskip

*Corresponding author: sangwoopark@snu.ac.kr
\end{flushleft}

\section*{Materials and Methods}

\subsection*{S1 Data}

The weekly surveillance data on pertussis cases across 252 municipalities (1st week of 2024 to the 44th week of 2025) were obtained from a publicly available website, the Infectious Disease Portal, by the Korea Disease Control and Prevention Agency \citep{kordata}.
Population size data as of September 2025 were obtained from publicly available website by the Ministry of the Interior and Safety \citep{kordata2}.
Vaccine coverage data were obtained from a publicly available website by the Korea Disease Control and Prevention Agency \citep{kordata3}.
Longitude and latitude data for each municipality were obtained from a publicly available GitHub repository \citep{github1}.
Finally, shape files used for constructing maps were obtained from a publicly available website \citep{shape}.

\subsection*{S2 Center of gravity}

We quantified the center of gravity to characterize variation in the mean timing of epidemic.
Typically, the center of gravity is calculated based on annual incidence for recurrent epidemics.
Since we are focused on analyzing a single outbreak, we use the entire time series of reported cases $C_t$ to computer the center of gravity for each municipality:
\begin{equation}
\textrm{Center of gravity}=\frac{\sum_{t=1}^{\tsub{t}{max}} C_t \times t}{\sum_{t=1}^{\tsub{t}{max}} C_t}.
\end{equation}
We then calculate the correlation coefficient between the center of gravity and the timing of introduction, defined as the first week when the number of reported cases is greater than 1 in 100,000.

\subsection*{S3 Spatial synchrony of reported cases}

We characterized the spatial synchrony of pertussis epidemic, which captures changes in pairwise correlation as a function of distance \citep{grenfell2001travelling}. 
We used logged values of reported cases between May 2024 and March 2025, excluding municipalities that had no reported cases ($n=250$).
Spatial synchrony was calculated using the \texttt{ncf} package in R \citep{ncf}.
We used 1,000 bootstraps to generate the median estimate and the corresponding 95\% confidence intervals.

\subsection*{S4 Effective reproduction number}

We calculated the effective reproduction number $\mathcal R(t)$ to characterize changes in pertussis transmission over time. 
First, we took the case time series $C_t$ and fitted a generalized additive model assuming a Poisson error to smooth the time series. 
Then, we estimated $\mathcal R(t)$ using the method of \citep{cori2013new}:
\begin{equation}
\mathcal R(t) = \frac{i(t)}{\sum_{k=1}^n i(t-k) g(k)},
\end{equation}
where $i(t)$ is the smoothed incidence and $n$ is the maximum length of generation interval in weeks (assumed to be 5 weeks).
The generation-interval distribution $g(k)$ is assumed to follow a gamma distribution with a mean of 9.47 days and a standard deviation of 6.22 days.
The reproduction number estimates can be unrealistically high when the number of infections is close to zero. 
Therefore, we only use estimates between May 2024 and March 2025 throughout the paper.
We truncated all $\mathcal{R}(t)$ estimates exceeding 5 to a maximum value of 5.

First, we compared correlation coefficients between $\mathcal R(t)$ estimates and those between logged cases across all pairwise municipality combinations (Figure 2B in the main text).
For this analysis, we only use data from municipalities with more than 400 total cases.
Then, we compared the spatial synchrony for $\mathcal R(t)$ estimates and synchrony for logged cases (Figure 2C in the main text).
For this analysis, we used $\mathcal R(t)$ estimates from all municipalities without any exclusion.

Finally, we fitted a generalized additive model to logged values of $\mathcal R(t)$ to quantify the effect of susceptible depletion and temporal variation in intrinsic transmission \citep{te2013driving,kissler2020projecting}:
\begin{equation}
\log \left(\mathcal R_{t,m}\right) = \alpha_m + \beta d_{t,m} + s(t),
\end{equation}
where $\mathcal R_{t,m}$ represents $\mathcal R(t)$ estimate at time $t$ in municipality $m$;
$\alpha_m$ represents the municipality-specific intercept term;
$\beta$ represents the effect of susceptible depletion;
$d_{t,m}$ represents the cumulative cases at time $t$ in municipality $m$;
and $s(t)$ represents the smooth term capturing temporal variation in transmission. 
For this analysis, we only use $\mathcal R(t)$ estimates from municipalities with more than 400 total cases.

\subsection*{S5 Transmission model}

Finally, we fitted a simple transmission model using Bayesian inference to test whether variation in introduction timing and susceptible dynamics alone can explain the observed epidemic patterns.
Specifically, we extended the SEIR model, which is commonly used for pertussis transmission \citep{rohani2010contact}, to allow for a joint estimation of a single, time-varying transmission term that is shared across all regions and a separate estimation of initial conditions and reporting rates.
To do so, we first discretized the SEIR model following the approach of \cite{he2010plug}:
\begin{align}
\beta (t) &= \mathcal R_0 (1-\exp(-\gamma\Delta t)) \delta(t)\\
\textrm{FOI}(t) &= \frac{\beta (t) I_m(t-\Delta t)}{N_m}\\
\Delta S_m(t) &= \left[1- \exp(-\textrm{FOI}(t) \Delta t)\right] S_m(t-\Delta t)\\
\Delta E_m(t) &= \left[1- \exp(-\sigma \Delta t)\right] E_m(t-\Delta t)\\
\Delta I_m(t) &= \left[1- \exp(-\gamma \Delta t)\right] I_m(t-\Delta t)\\
S_m(t) &= S(t-\Delta t) -\Delta S_m(t)\\
E_m(t) &= E(t-\Delta t) -\Delta E_m(t) + \Delta S_m(t)\\
I_m(t) &= I(t-\Delta t) -\Delta I_m(t) + \Delta E_m(t)
\end{align}
where $S_m(t)$, $E_m(t)$, and $I_m(t)$ represent the number of susceptible, exposed, and infectious individuals in municipality $m$ at time $t$, respectively;
$N_m$ represents the population size of municipality $m$;
$\Delta t$ represents the simulation time step, which assumed to be 1 week;
$\beta (t)$ represents the shared time-varying transmission rate;
$\mathcal R_0$ represents the shared basic reproduction number, assumed to equal 17 \citep{anderson1991infectious};
$\delta(t)$ represents the normalized time-varying transmission rate;
$\sigma$ represents the rate at which individuals develop infectiousness, which is assumed to be $\sigma=-\log(1-7/8)$ such that the mean latent period is 8 days;
and $\gamma$ represents the rate at which individuals recover, which is assumed to be $\sigma=-\log(1-7/15)$ such that the mean latent period is 15 days.

Here, we modeled changes in transmission using the $\delta(t)$ term, which is given a normal prior around 1:
\begin{equation}
\delta(t) \sim \mathrm{Normal}(1, 0.2).
\end{equation}
To allow for smooth variation in transmission, we also incorporated a random walk prior:
\begin{align}
\delta(t) &\sim \mathrm{Normal}(\delta(t-\Delta t), \sigma_\delta),\\
\sigma_\delta &\sim \textrm{Half-Normal}(0, 0.1),
\end{align}
where a half-normal prior on $\sigma_\delta$ constrains the smoothness of $\delta(t)$.

The municipality-specific observation process is modeled based on a Poisson distribution:
\begin{align}
\mathrm{cases}_{t,m} &\sim \mathrm{Poisson}(\rho_m \Delta S_m(t))\\
\rho_m &\sim \textrm{Uniform}(0, 1)
\end{align}
where $\mathrm{cases}_{t,m}$ represents the reported cases at time $t$ in municipality $m$;
$\rho_m$ represents the municipality-specific reporting rate;
and $\Delta S_m(t)$ represents the expected number of new infections between time $t-\Delta t$ and $t$.

Finally, we estimate a municipality-specific initial conditions by imposing the following priors:
\begin{align}
s_m(0) &\sim \textrm{Uniform}(0, 1),\\
i_m(0) &\sim \textrm{Half-Normal}(0, 0.001),
\end{align}
where $s_m(0)$ represents the initial fraction susceptible in municipality $m$, and $i_m(0)$ represents the initial fraction infected in municipality $m$.
Then, the initial conditions for the model is specified as follows:
\begin{align}
S_m(0) &= N_m s_m(0),\\
E_m(0) &= N_m i_m(0),\\
I_m(0) &= N_m i_m(0).
\end{align}
We simultaneously fit this model time series data between May 2024 and March 2025 from all municipalities with more than 400 total cases (a total of 42 municipalities) using a Bayesian inference software rstan \citep{carpenter2017stan}.

To assess the model fit, we compute R squared values for each municipality. 
This is done by calculating the squared value of the correlation coefficient between logged values of the observed cases and a posterior median of the logged predictions $\rho_m \Delta S_m(t)$.

Finally, we evaluate the impact of initial conditions on pertussis epidemic dynamics by varying $s(0)$ between $0.13$ and $0.23$ and $i(0)$ between $7\times 10^{-6}$ and $1.5\times 10^{-3}$.
For each simulation, we quantify the center of gravity.

\pagebreak

\bibliography{whooping-cough}

\end{document}
